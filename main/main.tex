\documentclass[letterpaper]{article}

\usepackage[margin=0.3in, top=0.1in]{geometry}
\usepackage{graphicx}

\renewcommand*\familydefault{\sfdefault}
\usepackage[T1]{fontenc}

\usepackage[dvipsnames]{xcolor}
\definecolor{ubcblue}{RGB}{0,10,62}
\definecolor{ubclightblue}{RGB}{171,223,234}

\usepackage{sectsty}
    \sectionfont{\color{cyan}}

\usepackage{array}

\usepackage{enumitem}
    \setlist[itemize]{noitemsep, topsep=-1.5ex}

\usepackage{titlesec}
    \titlespacing*{\section}
        {0pt}{-2pt}{-8pt}

\pagenumbering{gobble}

\setlength{\parindent}{0pt}

\renewcommand{\arraystretch}{0.1}

\newcommand{\sect}[1]{\section*{#1}
                        {\color{cyan}
                        \rule{\textwidth}{1pt}
                        \vspace{-1ex}}}

\begin{document}
    \begin{flushright}
        \includegraphics[width=0.4\textwidth]{ScienceCo-opLogo_UBC_H.eps}
    \end{flushright}

    \vspace{-2ex}
    \begin{minipage}{\dimexpr(\textwidth-56pt)}
        \begin{flushright}
        {\color{ubclightblue}\rule{\textwidth}{0.5pt}}

        {\small\color{ubcblue}
          T: 604.822.9677
        | F: 604.822.9677
        | science.coop@ubc.ca
        | www.sciencecoop.ubc.ca}
        \end{flushright}
    \end{minipage}

    {\Huge
    \textbf{Jason Ngo}}

    \vspace{1ex}
    {\large
    Computer Science Major @ UBC}

    {\small
      +1 587-890-5411
    | work@jasonn.dev
    | github.com/Green-Avocado
    | www.jasonn.dev
    }
    \vspace{1ex}

    \sect{Skills Summary}

        \begin{tabular}{p{0.18\textwidth}
                        p{0.78\textwidth}}
            \textbf{Application Security} &
                Buffer overflow,
                Format-string exploits,
                Return-oriented programming,
                Use-after-free
                \\
                \\
            \textbf{Web Security} &
                SQL injection,
                Cross-site scripting,
                Template injection,
                Local file inclusion,
                Prototype pollution
                \\
                \\
            \textbf{Systems development} &
                Rust,
                x86 Assembly,
                C / C++,
                Java
                \\
                \\
            \textbf{Web development} &
                NodeJS,
                REST APIs,
                NGINX,
                Google Firebase
                \\
                \\
            \textbf{System administration} &
                Linux,
                Docker,
                SQL
                \\
                \\

                \\\\\\\\
        \end{tabular}

    \sect{Work Experience}

        \begin{tabular}{p{0.18\textwidth} p{0.78\textwidth}}
            \textbf{2020/04 - 2022/02} & \textbf{Freelance Software Development} \\
            & \emph{Commissioned by clients for various projects. Some examples include:} \\
            & \begin{itemize}
                \item \underline{Transactions database} ---
                    Designed proof-of-concepts for database solutions using Firebase Realtime Database,
                    MySQL, and Google Drive APIs.
                \item \underline{Mosque timetable} ---
                    Developed a web application to read data from a CSV file and display prayer times
                    using HTML, CSS, and JavaScript.
                \item \underline{covidping.com} ---
                    Wrote scripts to load current COVID-19 statistics into Google Sheets and send emails
                    to a list of subscribers for to notify users of COVID-19 statistics in their state.
            \end{itemize}
        \end{tabular}

    \sect{Technical Extracurriculars}

        \begin{tabular}{p{0.18\textwidth} p{0.78\textwidth}}
            \textbf{2019/09 - Present} & \textbf{Capture The Flag Competitions} \\
            & \emph{https://github.com/Green-Avocado/CTF} \\
            & \begin{itemize}
                \item Reverse engineered binaries without symbols using Ghidra and Radare2.
                \item Performed dynamic analysis and debugged exploits using GDB.
                \item Identified vulnerabilities in binary applications and web services.
                \item Defeated common exploit mitigations such as position independent executables,
                    address-space layout randomization, stack canaries, and relocation read-only.
                \item Created writeups to explain vulnerabilities and exploit techniques used in each
                    challenge.
            \end{itemize}
        \end{tabular}

        \begin{tabular}{p{0.18\textwidth} p{0.78\textwidth}}
            \textbf{2017/09 - 2020/02} & \textbf{Vex Robotics Club} \\
            & \emph{https://github.com/Green-Avocado/3388D-vex-robotics-edr-2020} \\
            & \begin{itemize}
                \item Wrote firmware in C++ which used the Vex API to receive instructions from a
                    controller.
                \item Used feedback from sensor data to guide autonomous routines and aid user control.
                \item Created a user interface for the controller display screen to configure the robot at
                    runtime.
                \item Our team won a programming award and we were invited to compete in the international
                    event.
            \end{itemize}
        \end{tabular}

    \sect{Hackathon Projects}

        \begin{tabular}{p{0.18\textwidth} p{0.78\textwidth}}
            \textbf{2022/01} & \textbf{Language Exchange} \\
            & \emph{https://github.com/Green-Avocado/Language-Exchange} \\
            & \begin{itemize}
                \item In a team of 4, created a website using React and NodeJS for connecting language
                    students with complementary strengths and goals.
                \item Used Google Firebase to set up a database and user authentication, allowing users to
                    log in using their existing Google accounts.
                \item Stored user data in the Firebase real-time database, which could be queried and
                    filtered to match users according to their learning goals.
            \end{itemize}
        \end{tabular}

        \begin{tabular}{p{0.18\textwidth} p{0.78\textwidth}}
            \textbf{2021/11} & \textbf{Speak-able} \\
            & \emph{https://devpost.com/software/speak-able-inclusive-unconferencing} \\
            & \begin{itemize}
                \item In a team of 4, created a website for encouraging inclusivity in
                    participant-driven meetings.
                \item Used NodeJS and Express to implement a RESTful API to interact with the webpage,
                    allowing users to submit new topics, vote for existing topics, and view the number
                    of votes for each topic.
            \end{itemize}
        \end{tabular}

    \pagebreak
    {\small
        +1 587-890-5411
        \hfill
        work@jasonn.dev
        \hfill
        github.com/Green-Avocado
        \hfill
        www.jasonn.dev
    }
    \vspace{-8pt}

    \rule{\textwidth}{0.1pt}

    \vspace{0.2in}

        \begin{tabular}{p{0.18\textwidth} p{0.78\textwidth}}
            \textbf{2020/08} & \textbf{Study Tinder} \\
            & \emph{https://devpost.com/software/study-tinder} \\
            & \begin{itemize}
                \item In a team of 2, created a website for helping students connect and study while
                    social distancing.
                \item Used Google Firebase to set up hosting, a database, and user authentication.
                \item Matched students with complementary interests by using JavaScript to query the
                    Firestore database.
            \end{itemize}
        \end{tabular}

        \begin{tabular}{p{0.18\textwidth} p{0.78\textwidth}}
            \textbf{2020/08} & \textbf{BikePath} \\
            & \emph{https://devpost.com/software/bikepath-dkpstx} \\
            & \begin{itemize}
                \item In a team of 3, created a website to help users find alternative locations that
                    would permit eco-friendly alternatives to driving, such as walking or biking.
                \item Used the Google Maps API to render a map view of selected locations on the webpage.
                \item Used Python to interact with the Google Places API to retrieve, parse, and
                    interpret data.
            \end{itemize}
        \end{tabular}

        \begin{tabular}{p{0.18\textwidth} p{0.78\textwidth}}
            \textbf{2020/08} & \textbf{COVID Wait} \\
            & \emph{https://devpost.com/software/covid-wait} \\
            & \begin{itemize}
                \item In a team of 5, created a website to help users avoid highly populated areas and
                    reduce the risk of exposure to COVID-19.
                \item Used Python to develop a server-side application to retrieve data from the Google
                    Maps API.
                \item Implemented a RESTful API to allow clients to query data from the server.
            \end{itemize}
        \end{tabular}

    \sect{Cybersecurity Projects}

        \begin{tabular}{p{0.18\textwidth} p{0.78\textwidth}}
            \textbf{2021/03 - Present} & \textbf{pwndocker} \\
            & \emph{https://github.com/Green-Avocado/pwndocker} \\
            & \begin{itemize}
                \item Wrote a minimal program in C to create symbolic links without standard libraries.
                \item Used Docker and gdbserver to create an environment for debugging exploits against
                    applications using different versions of the GNU C Library.
                \item The project became a go-to tool for CTF challenges involving binary exploitation.
            \end{itemize}
        \end{tabular}

        \begin{tabular}{p{0.18\textwidth} p{0.78\textwidth}}
            \textbf{2022/02} & \textbf{BBY Stealer Malware Analysis} \\
            & \emph{https://github.com/Green-Avocado/bbystealer-malware-analysis} \\
            & \begin{itemize}
                \item Performed dynamic analysis using Wireshark to identify external connections and Windows
                    filesystem auditing to identify files read or modified.
                \item Reverse engineered JavaScript code that was obfuscated using obfuscator.io and
                    packaged as a Windows executable using nexe.
                \item Helped victims with incident response by identifying compromised credentials and
                    modified files.
            \end{itemize}
        \end{tabular}

        \begin{tabular}{p{0.18\textwidth} p{0.78\textwidth}}
            \textbf{2021/10 - 2022/01} & \textbf{UBC MapleCTF} \\
            & \emph{https://github.com/ubcctf/maple-ctf-ubc-2022} \\
            & \begin{itemize}
                \item Wrote challenges in C with intentional vulnerabilities to teach binary exploitation
                    techniques.
                \item Used Docker to containerize challenges so they could be deployed through Kubernetes.
                \item Helped beginners by running demonstrations at a workshop and answering questions
                    related to binary exploitation and reverse engineering.
            \end{itemize}
        \end{tabular}

        \begin{tabular}{p{0.18\textwidth} p{0.78\textwidth}}
            \textbf{2021/09 - 2021/12} & \textbf{EasyROP} \\
            & \emph{https://github.com/Green-Avocado/EasyROP} \\
            & \begin{itemize}
                \item Wrote a program in Java using principles of object-oriented design to automate
                    the process of writing scripts for binary exploitation.
                \item The project began as a command-line application and later included a graphical user
                    interface which was developed using Java Swing.
                \item Return-oriented programming payloads could be saved as a local JSON file and
                    reloaded.
            \end{itemize}
        \end{tabular}

    \sect{Personal Projects}

        \begin{tabular}{p{0.18\textwidth} p{0.78\textwidth}}
            \textbf{2020/12 - Present} & \textbf{website} \\
            & \emph{https://github.com/Green-Avocado/website} \\
            & \begin{itemize}
                \item Used NodeJS with Express to serve web pages which are generated using a templating
                    engine.
                \item Used NGINX to secure connections using TLS and forward HTTP requests to internal
                    services.
                \item Used Docker to containerize services, allowing each to be modified and restarted
                    independently.
                \item Tested the website and scanned for vulnerabilities using continuous integration.
            \end{itemize}
        \end{tabular}

        \begin{tabular}{p{0.18\textwidth} p{0.78\textwidth}}
            \textbf{2021/11 - 2022/01} & \textbf{atom-ide-rust} \\
            & \emph{https://github.com/rust-lang/atom-ide-rust} \\
            & \begin{itemize}
                \item Contributed to an open source plugin for integrating rust-analyzer into the Atom
                    text editor.
                \item Used NodeJS to read config files, parse JSON data, and interface with a language
                    server.
                \item Wrote documentation using markdown to explain the usage of the plugin with examples.
                \item The plugin has been downloaded over 164 000 times by developers programming with
                    Rust.
            \end{itemize}
        \end{tabular}

        \begin{tabular}{p{0.18\textwidth} p{0.78\textwidth}}
            \textbf{2021/12} & \textbf{discord-balance-tracker} \\
            & \emph{https://github.com/Green-Avocado/discord-balance-tracker} \\
            & \begin{itemize}
                \item Wrote a Rust application to provide a convenient way to track balances throught
                    Discord.
                \item Used asynchronous programming to send and receive interactions through the Discord
                    API.
            \end{itemize}
        \end{tabular}

        \begin{tabular}{p{0.18\textwidth} p{0.78\textwidth}}
            \textbf{2021/03} & \textbf{Etwahl} \\
            & \emph{https://github.com/Green-Avocado/Etwahl} \\
            & \begin{itemize}
                \item Wrote a program in C++ to bind signals from an electronic piano to simulated
                    keyboard events.
                \item Received MIDI signals over a USB connection using the open source RtMidi library.
                \item Used CMake, X11 libraries, and Windows libraries to develop a multi-platform
                    application.
            \end{itemize}
        \end{tabular}

    \sect{Awards}

        \begin{tabular}{p{0.18\textwidth} p{0.78\textwidth}}
            \textbf{2022/01} & CyberSci Vancouver Regionals - First Place \\
            \textbf{2020/03} & Vex EDR Alberta Provincial Tournament - Think Award \\

            \\\\\\\\
        \end{tabular}

    \sect{Education}

        \begin{tabular}{p{0.18\textwidth} p{0.78\textwidth}}
            \textbf{2020/09 - 2024/04} & \large\textbf{Bachelor of Science, Major in Computer Science} \\
            & \emph{University of British Columbia, Vancouver, BC} \\
        \end{tabular}

\end{document}

